\section{Arquitectura de Módulos}
Un módulo de kernel es un fragmento de código objeto que se inserta o retira en tiempo de ejecución.  
\begin{itemize}
  \item Permite mantener el kernel liviano: solo se cargan drivers necesarios.
  \item Mejora la depuración: un módulo defectuoso puede descargarse sin reiniciar.
  \item Ejemplos típicos: drivers de bus (PCI, USB, I²C) y drivers de dispositivo.
\end{itemize}

\section{Drivers, Device Drivers y Bus Drivers}

En el kernel de Linux se distinguen tres tipos de controladores:

\begin{description}
  \item[Driver:] Componente genérico que ofrece servicios a subsistemas lógicos o hardware, controlando recursos diversos.
  \item[Device driver:] Gestiona un periférico concreto (disco, cámara, puerto serie), traduciendo operaciones estándar (read, write, ioctl) en accesos al dispositivo.
  \item[Bus driver:] Implementa el protocolo de un bus de comunicación (USB, PCI, I²C), proveyendo la capa de bajo nivel para que los \emph{device drivers} interactúen con el hardware.
\end{description}

\noindent Todos los \emph{device drivers} utilizan la interfaz que provee el \emph{bus driver} correspondiente, garantizando una arquitectura modular y una abstracción uniforme del hardware.


\section{Clasificación de Controladores}
Según el tipo de I/O:
\begin{enumerate}
  \item \textbf{Character (CDD):} flujos byte a byte, sin estructura de bloques. Ej.: teclados, puertos seriales.
  \item \textbf{Block:} bloques direccionables, acceso aleatorio. Ej.: discos, SSD.
  \item \textbf{Network:} paquetes o tramas, siguen protocolos de red.
\end{enumerate}

\section{Clasificación de los controladores de dispositivo}

Existen varias maneras de agrupar los \emph{device drivers}. La clasificación más común se basa en el tipo de datos o el modo de acceso que cada controlador ofrece al sistema operativo. Tradicionalmente, se distinguen tres categorías principales:

\begin{description}
  \item[1. Drivers orientados a bytes (\texttt{Character}):]  
    Gestionan flujos de datos secuenciales, un byte tras otro, sin estructuras de bloque.  
    \begin{itemize}
      \item No permiten posicionamiento aleatorio: la lectura o escritura ocurre siempre de forma continua.  
      \item No suelen disponer de buffering interno significativo.  
      \item Ejemplos: teclados (\texttt{/dev/tty*}), ratones, terminales de texto, impresoras paralelas, puertos serie (COM, UART), algunos sensores y dispositivos de E/S simples.  
    \end{itemize}

  \item[2. Drivers orientados a bloques (\texttt{Block}):]  
    Operan sobre unidades de datos en bloques de tamaño fijo, permitiendo acceso aleatorio a cada bloque.  
    \begin{itemize}
      \item El sistema de archivos los requiere para montar volúmenes de almacenamiento.  
      \item Se identifican con la letra \texttt{b} en listados de \texttt{/dev} (por ejemplo, \texttt{/dev/sda} aparece como \texttt{brw-r--r--}).  
      \item Ejemplos: discos duros (HDD, SSD), unidades ópticas, tarjetas de memoria.  
    \end{itemize}

  \item[3. Drivers orientados a paquetes (\texttt{Network}):]  
    Manejan la transmisión y recepción de tramas o paquetes según protocolos de red.  
    \begin{itemize}
      \item Aunque internamente usan flujos de bytes, los exponen como interfaces de red gestionadas fuera de \texttt{/dev} (normalmente en \texttt{/sys/class/net} o mediante herramientas \texttt{ip}/\texttt{ifconfig}).  
      \item Ejemplos: tarjetas Ethernet, adaptadores WiFi, dispositivos Bluetooth.  
    \end{itemize}
\end{description}

\medskip

Además de estas tres, existen casos especiales:
\begin{itemize}
  \item \emph{Dispositivos virtuales} (por ejemplo \texttt{/dev/null}, \texttt{/dev/zero}), que se comportan como CDD pero no corresponden a hardware físico.
  \item \emph{Controladores de subsistemas lógicos}, como sistemas de archivos en memoria (\texttt{tmpfs}) o drivers de cifrado, que no encajan estrictamente en las categorías clásicas.
\end{itemize}

En un sistema típico de Linux, los drivers de carácter son los más numerosos: casi todo lo que no es almacenamiento masivo (\emph{block}) ni interfaz de red (\emph{network}) se implementa como CDD. Esta preponderancia responde a la gran diversidad de periféricos sencillos (sensores, puertos serie, consolas, dispositivos de audio básicos, etc.) que requieren comunicación un byte a la vez.

Gracias a esta taxonomía, el kernel puede organizar mejor su espacio de nombres de dispositivos, asignar adecuadamente números major/minor y delegar cada tipo de acceso al subsistema correspondiente, optimizando el rendimiento y la mantenibilidad del sistema.



