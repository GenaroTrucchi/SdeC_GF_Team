\section{Conclusiones}

El presente trabajo práctico permitió recorrer desde los conceptos teóricos de la arquitectura \textbf{UEFI} hasta la implementación y prueba de un \emph{bootloader} propio, revelando la interacción detallada entre firmware, herramientas de \emph{toolchain} y hardware.

\begin{itemize}
  \item \textbf{UEFI como reemplazo de BIOS}. Comprendimos que UEFI no es una simple evolución sino una plataforma firmware escrita en~C, capaz de operar en 32~/ 64~bits, con servicios de arranque y ejecución en tiempo de \emph{runtime}, soporte GPT, \emph{Secure Boot} y controladores integrados. Aprender a invocar sus tablas (\texttt{BootServices} y \texttt{RuntimeServices}) habilita el desarrollo de utilidades de muy bajo nivel que conviven con el sistema operativo.
  \item \textbf{Superficie de ataque y riesgos}. Casos como \emph{LoJax}, \emph{ThinkPwn} o \emph{LogoFAIL} evidencian que vulnerar el firmware permite \emph{bootkits} persistentes aun tras reinstalar el SO. La seguridad del arranque exige tanto configuraciones correctas (protección de la flash) como la aplicación oportuna de parches.
  \item \textbf{CSME / ME y MEBx}. Las funciones de gestión remota (AMT) y arranque verificado (Boot Guard) ofrecen ventajas, pero el motor autónomo de Intel sigue siendo una \emph{caja negra} con fallos críticos publicados. Conocer MEBx resulta clave para equilibrar administración y exposición al riesgo.
  \item \textbf{coreboot: firmware libre}. coreboot demuestra que es posible reemplazar firmware propietario por una solución mínima y auditada, con arranques más veloces y menor superficie de ataque, manteniendo la flexibilidad gracias a los \emph{payloads} (SeaBIOS, TianoCore, LinuxBoot, etc.).
  \item \textbf{Herramientas de \emph{toolchain}}. Revisar el \texttt{linker} y la convención histórica de 0x7C00 mostró que los scripts de enlace son tan cruciales como el código; si el \emph{linker} desconoce la dirección final, los desplazamientos fallan y el arranque se aborta.
  \item \textbf{Bootloader ``HELLO''}. Desde ensamblar 512~bytes y firmarlos con \texttt{0xAA55}, hasta probar en QEMU y grabar en un pendrive, cerramos el ciclo completo: escribir, enlazar, testear y ejecutar código que arranca sin sistema operativo.
\end{itemize}

Además, el desafío final permitió poner en práctica conceptos clave de la transición al Modo Protegido:

\begin{itemize}
  \item \textbf{Transición Manual a Modo Protegido.} El diseño de un bootloader completamente manual, sin macros ni abstracciones, obligó a comprender en profundidad el flujo de activación del PE (Protection Enable), la importancia del salto largo posterior (\texttt{ljmp}), y la configuración explícita de los registros de segmento y la pila en 32 bits.
  \item \textbf{Uso Correcto de ELF y Linker Scripts.} La necesidad de ensamblar primero a ELF y luego enlazar usando un \texttt{link.ld} a 0x7C00 destacó el rol crítico del linker en sistemas bare-metal, resolviendo problemas típicos de formatos binarios directos (\texttt{-f bin}) como desincronizaciones y referencias incorrectas.
  \item \textbf{Segmentación y Protección.} Implementar descriptores diferenciados de código, datos R/W y datos R/O permitió experimentar directamente con la protección por hardware: definir un segmento de sólo lectura y verificar mediante ejecución que la CPU detecta y bloquea accesos inválidos.
  \item \textbf{Verificación de Protección por Hardware.} El intento de escribir en un segmento R/O, seguido de la observación de excepciones de Protección General (GP) y triple fallo en emuladores precisos como Bochs y QEMU-KVM, demostró empíricamente cómo el hardware asegura el cumplimiento de permisos segmentados, incluso antes de introducir mecanismos de paginación.
\end{itemize}

Estos ejercicios finales consolidaron no sólo la teoría del modelo de segmentación x86 y del proceso de arranque, sino también la práctica detallada de debugging en entornos reales (GDB, QEMU, Bochs), el entendimiento del rol de los selectores en modo protegido y la importancia de diseñar firmwares mínimos pero seguros.
