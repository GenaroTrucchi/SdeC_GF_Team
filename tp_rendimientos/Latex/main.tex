\documentclass[a4paper,12pt]{article}

% Codificación y lenguaje
\usepackage[utf8]{inputenc}
\usepackage[T1]{fontenc}
\usepackage[spanish]{babel}
\usepackage[utf8]{inputenc}

% Paquetes esenciales
\usepackage{graphicx}
\usepackage{geometry}
\usepackage{datetime}
\usepackage{titlesec}
\usepackage{fancyhdr}
\usepackage{parskip}
\usepackage{array}
\usepackage{float}

% Paquetes para código fuente
\usepackage{listings}
\usepackage{xcolor}

% Configuración del estilo de listings para código Arduino
\lstdefinestyle{arduino}{
  language=C++,
  basicstyle=\ttfamily\footnotesize,
  keywordstyle=\color{blue},
  commentstyle=\color{gray},
  stringstyle=\color{red},
  numbers=left,
  numberstyle=\tiny,
  stepnumber=1,
  numbersep=5pt,
  backgroundcolor=\color{white},
  frame=single,
  breaklines=true,
  captionpos=b
}
\lstset{style=arduino}

% Márgenes del documento
\geometry{left=3cm, right=2.5cm, top=3cm, bottom=3cm}

\begin{document}

% Incluir carátula (asegurate de que caratula.tex exista)
\begin{titlepage}
    \begin{center}
        {\LARGE \textbf{Universidad Nacional de Córdoba}}\\[1.5cm]

        \includegraphics[scale=0.4]{img/logo2.png}\\[1.5cm]

        {\large Facultad de Ciencias Exactas, Físicas y Naturales}\\
        {\large Escuela de Electrónica}\\[1cm]

        \rule{\linewidth}{0.5mm}\\[0.4cm]
        {\Large \textbf{Cátedra de Electrónica Analógica 3}}\\[0.3cm]
        {\LARGE \textbf{Trabajo de Laboratorio 1}}\\[0.3cm]
        \rule{\linewidth}{0.5mm}\\[1cm]

        \begin{flushleft}
        {\large 
            \textbf{Profesor Titular:} Ing. Rodrigo Gabriel Bruni\\
            \textbf{Profesor Adjunto:} -\\[0.5cm]
            \textbf{Integrantes:}\\
            Trucchi, Genaro\\
        }
        \end{flushleft}

        \vfill

        {\large \today}
    \end{center}
\end{titlepage}


% Contenido del informe
\section{Resumen}
Aquí va el resumen del trabajo.

\section{Introducción}
Aquí va la introducción al informe.

\newpage

\section{Resultados}
Aquí se detallan los resultados del laboratorio.

\subsection{Resultados PC Agustin}
Mateo y Genaro son muy buenos compañeros de equipo!

\newpage

\subsection{Resultados PC Mateo}
\section{Benchmark 1}
Hola como estas

\newpage

\subsection{Resultados PC Genaro}
\subsubsection*{1. Uso cotidiano de la PC}
La PC se utiliza principalmente para desarrollo de software y ejecución de simulaciones. Los programas y entornos utilizados incluyen:
\begin{itemize}
    \item \textbf{Desarrollo de software:} Visual Studio Code, PyCharm, Eclipse.
    \item \textbf{Simulaciones:} MATLAB, Simulink, ANSYS.
    \item \textbf{Otros:} Navegación web con múltiples pestañas, edición de documentos en LaTeX.
\end{itemize}

\subsubsection*{2. Tareas y benchmarks representativos}

A continuación se presenta una tabla con algunas tareas frecuentes relacionadas con simulaciones y desarrollo de código, junto con el benchmark que mejor representa cada una.

\begin{center}
\begin{tabular}{|p{7cm}|p{7cm}|}
\hline
\textbf{Tarea} & \textbf{Benchmark representativo} \\
\hline
Compilar proyectos grandes en C/C++ & Geekbench o Cinebench (CPU multi-core) \\
\hline
Ejecución de simulaciones en MATLAB/Simulink & SPEC CPU2017 \\
\hline
Desarrollo de software en entornos pesados (e.g., PyCharm, Eclipse) & PassMark CPU \\
\hline
Análisis de datos y cálculos intensivos en Python & Geekbench (CPU single-core) \\
\hline
\end{tabular}
\end{center}

\subsubsection*{3. Reflexión final}
En términos generales, la PC cumple con los requerimientos diarios para el desarrollo de software y la ejecución de simulaciones. Sin embargo, se ha identificado que el rendimiento de la CPU puede convertirse en un cuello de botella durante la ejecución de simulaciones complejas y la compilación de proyectos grandes. 

Existe interés en evaluar y mejorar el rendimiento en estas tareas específicas, por lo que se considerará realizar actualizaciones de hardware, como la incorporación de un procesador más potente o la ampliación de la memoria RAM, para optimizar el rendimiento.


\newpage

\subsection{Comparación de CPUs}

\subsubsection*{Comparación de rendimiento: compilación del kernel de Linux}

En esta sección se analiza el rendimiento de tres procesadores modernos al compilar el kernel de Linux, una tarea intensiva y altamente paralelizable que representa una carga exigente para la CPU y el subsistema de memoria. Aunque el equipo bajo análisis utiliza Windows, la compilación del kernel de Linux es una prueba objetiva de rendimiento bruto útil para comparar CPUs.

\subsubsection*{Procesadores evaluados}

\begin{itemize}
  \item \textbf{Intel Core i5-13600K} (14 núcleos / 20 hilos, arquitectura híbrida Alder Lake)
  \item \textbf{AMD Ryzen 9 5900X} (12 núcleos / 24 hilos, arquitectura Zen 3)
  \item \textbf{AMD Ryzen 9 7950X} (16 núcleos / 32 hilos, arquitectura Zen 4)
\end{itemize}

\subsubsection*{Rendimiento bruto en compilación}

La prueba consiste en compilar el kernel de Linux con configuración por defecto, según benchmarks públicos disponibles en \textit{Phoronix} y \textit{OpenBenchmarking.org}.

\begin{itemize}
  \item El \textbf{Ryzen 9 7950X} logra tiempos de compilación extraordinariamente bajos, en torno a \textbf{40 segundos}.
  \item El \textbf{Ryzen 9 5900X}, con menos núcleos y menor IPC, se estima que tarda entre \textbf{80 a 90 segundos}, lo cual representa un rendimiento 50\% inferior.
  \item El \textbf{Core i5-13600K} también ofrece buen rendimiento gracias a sus 20 hilos, pero queda relegado frente a los Ryzen de gama entusiasta. Se estima que su tiempo ronda los \textbf{90 segundos o más}.
\end{itemize}

\subsubsection*{Aceleración del Ryzen 9 7950X}

Comparando los tiempos de compilación, se puede calcular la aceleración relativa del Ryzen 9 7950X frente a los otros dos modelos:

\begin{itemize}
  \item \textbf{Frente al Core i5-13600K:} El 7950X es al menos \textbf{40\% más rápido en promedio general}, y en cargas puramente multihilo como la compilación puede llegar a ser hasta \textbf{100\% más rápido} (el doble de rápido).
  
  \item \textbf{Frente al Ryzen 9 5900X:} La ventaja del 7950X oscila entre \textbf{60\% y 80\% de mayor rendimiento}. Tareas que el 5900X ejecuta en 1.5 minutos, el 7950X las resuelve en menos de 1 minuto.
\end{itemize}

Esta mejora se debe a su mayor cantidad de núcleos, mayor IPC y mayor frecuencia base. Para desarrolladores que compilan grandes proyectos frecuentemente, esta diferencia puede traducirse en horas de trabajo ahorradas cada semana.

\subsubsection*{Resumen comparativo}

\begin{center}
\begin{tabular}{|p{4.5cm}|c|c|}
\hline
\textbf{Procesador} & \textbf{Tiempo estimado de compilación} & \textbf{Aceleración del 7950X} \\
\hline
Intel Core i5-13600K & 90+ s & 2× más rápido \\
\hline
AMD Ryzen 9 5900X & 80–90 s & 1.6–2× más rápido \\
\hline
AMD Ryzen 9 7950X & \textbf{~40 s} & --- \\
\hline
\end{tabular}
\end{center}

\subsubsection*{Conclusión}

El AMD Ryzen 9 7950X se posiciona como el procesador más potente de este grupo para la tarea de compilación del kernel de Linux. Gracias a sus 16 núcleos de alto rendimiento y la mejora de IPC de la arquitectura Zen 4, supera ampliamente tanto al Ryzen 9 5900X como al Intel Core i5-13600K. Aunque este último ofrece una excelente relación rendimiento-precio, para cargas intensivas y recurrentes como la compilación masiva de software, los procesadores Ryzen de gama entusiasta marcan una diferencia sustancial.


\newpage

\section{Conclusiones}
Conclusiones del trabajo realizado.

\end{document}
