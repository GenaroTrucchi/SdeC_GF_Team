\subsubsection*{Comparación de rendimiento: compilación del kernel de Linux}

En esta sección se analiza el rendimiento de tres procesadores modernos al compilar el kernel de Linux, una tarea intensiva y altamente paralelizable que representa una carga exigente para la CPU y el subsistema de memoria. Aunque el equipo bajo análisis utiliza Windows, la compilación del kernel de Linux es una prueba objetiva de rendimiento bruto útil para comparar CPUs.

\subsubsection*{Procesadores evaluados}

\begin{itemize}
  \item \textbf{Intel Core i5-13600K} (14 núcleos / 20 hilos, arquitectura híbrida Alder Lake)
  \item \textbf{AMD Ryzen 9 5900X} (12 núcleos / 24 hilos, arquitectura Zen 3)
  \item \textbf{AMD Ryzen 9 7950X} (16 núcleos / 32 hilos, arquitectura Zen 4)
\end{itemize}

\subsubsection*{Rendimiento bruto en compilación}

La prueba consiste en compilar el kernel de Linux con configuración por defecto, según benchmarks públicos disponibles en \textit{Phoronix} y \textit{OpenBenchmarking.org}.

\begin{itemize}
  \item El \textbf{Ryzen 9 7950X} logra tiempos de compilación extraordinariamente bajos, en torno a \textbf{40 segundos}.
  \item El \textbf{Ryzen 9 5900X}, con menos núcleos y menor IPC, se estima que tarda entre \textbf{80 a 90 segundos}, lo cual representa un rendimiento 50\% inferior.
  \item El \textbf{Core i5-13600K} también ofrece buen rendimiento gracias a sus 20 hilos, pero queda relegado frente a los Ryzen de gama entusiasta. Se estima que su tiempo ronda los \textbf{90 segundos o más}.
\end{itemize}

\subsubsection*{Aceleración del Ryzen 9 7950X}

Comparando los tiempos de compilación, se puede calcular la aceleración relativa del Ryzen 9 7950X frente a los otros dos modelos:

\begin{itemize}
  \item \textbf{Frente al Core i5-13600K:} El 7950X es al menos \textbf{40\% más rápido en promedio general}, y en cargas puramente multihilo como la compilación puede llegar a ser hasta \textbf{100\% más rápido} (el doble de rápido).
  \item \textbf{Frente al Ryzen 9 5900X:} La ventaja del 7950X oscila entre \textbf{60\% y 80\% de mayor rendimiento}. Tareas que el 5900X ejecuta en 1.5 minutos, el 7950X las resuelve en menos de 1 minuto.
\end{itemize}

Esta mejora se debe a su mayor cantidad de núcleos, mayor IPC y mayor frecuencia base. Para desarrolladores que compilan grandes proyectos frecuentemente, esta diferencia puede traducirse en horas de trabajo ahorradas cada semana.

\subsubsection*{Resumen comparativo}

\begin{center}
\begin{tabular}{|p{4.5cm}|c|c|}
\hline
\textbf{Procesador} & \textbf{Tiempo estimado de compilación} & \textbf{Aceleración del 7950X} \\
\hline
Intel Core i5-13600K & 90+ s & 2× más rápido \\
\hline
AMD Ryzen 9 5900X & 80–90 s & 1.6–2× más rápido \\
\hline
AMD Ryzen 9 7950X & \textbf{~40 s} & --- \\
\hline
\end{tabular}
\end{center}

\subsubsection*{Conclusión}

El AMD Ryzen 9 7950X se posiciona como el procesador más potente de este grupo para la tarea de compilación del kernel de Linux. Gracias a sus 16 núcleos de alto rendimiento y la mejora de IPC de la arquitectura Zen 4, supera ampliamente tanto al Ryzen 9 5900X como al Intel Core i5-13600K. Aunque este último ofrece una excelente relación rendimiento-precio, para cargas intensivas y recurrentes como la compilación masiva de software, los procesadores Ryzen de gama entusiasta marcan una diferencia sustancial.

\bigskip

\subsubsection*{Prueba de rendimiento en la ESP32}

Para evaluar el impacto de la frecuencia del reloj en el rendimiento de la ESP32, utilizamos la plataforma PlatformIO dentro de VSCode por su simplicidad. Luego de crear el proyecto y configurar la placa, se utilizó el siguiente código:

\lstset{style=arduino}

\subsubsection*{Código de prueba de frecuencia en ESP32}

\begin{lstlisting}[language=C++, caption={Código para medir el tiempo de ejecución en ESP32}, label=lst:esp32_code]
#include <Arduino.h>

void setup() {
  Serial.begin(115200);
  delay(1000); // Espera para la inicialización del Serial

  // Cambiar la frecuencia de la CPU a 80MHz (opcional)
  setCpuFrequencyMhz(80); // O 160 o 240
  Serial.print("Frecuencia de CPU: ");
  Serial.print(getCpuFrequencyMhz());
  Serial.println(" MHz");

  // Medir el tiempo de ejecución del bucle
  unsigned long startTime = millis();

  // Bucle for con sumas de enteros
  int sumInt = 0;
  for (int i = 0; i < 1000000; i++) {
    sumInt += i;
  }

  // Bucle for con sumas de floats
  float sumFloat = 0.0;
  for (float f = 0.0; f < 1000000.0; f += 1.0) {
    sumFloat += f;
  }

  unsigned long endTime = millis();
  unsigned long duration = endTime - startTime;

  Serial.print("Suma de enteros: ");
  Serial.println(sumInt);
  Serial.print("Suma de floats: ");
  Serial.println(sumFloat);
  Serial.print("Tiempo de ejecución: ");
  Serial.print(duration);
  Serial.println(" ms");
}

void loop() {
  // No hay nada en el bucle principal
}
\end{lstlisting}

Este código mide el tiempo de ejecución de un millón de sumas enteras y de punto flotante, permitiendo comparar el rendimiento de la ESP32 a distintas frecuencias de reloj.

\subsubsection*{Resultados experimentales}

Con una frecuencia de 80 MHz, obtuvimos un tiempo total de ejecución de aproximadamente **4987 ms (4.987 segundos)**.

\begin{figure}[H]
    \centering
    \includegraphics[width=0.7\linewidth]{img/ESP32Test80MG.png}
    \caption{Prueba a 80 MHz}
    \label{fig:esp32_80mhz}
\end{figure}

Al repetir la prueba con una frecuencia de 160 MHz, el tiempo se redujo a **2465 ms (2.465 segundos)**.

\begin{figure}[H]
    \centering
    \includegraphics[width=0.7\linewidth]{img/ESP32Test160Mhz.png}
    \caption{Prueba a 160 MHz}
    \label{fig:esp32_160mhz}
\end{figure}

\subsubsection*{Conclusiones}

Comparando los tiempos de ejecución en la ESP32 con distintas frecuencias:

\begin{itemize}
  \item \textbf{Reducción del tiempo de ejecución:} Al duplicar la frecuencia del reloj de 80 MHz a 160 MHz, el tiempo de ejecución se reduce significativamente.
  \item \textbf{Relación no perfectamente lineal:} Aunque el rendimiento mejora, la relación no es exactamente 2:1 debido a factores como latencia de memoria y eficiencia interna del procesador.
  \item \textbf{Relevancia práctica:} Para tareas que requieren procesamiento intensivo, aumentar la frecuencia puede ser beneficioso, aunque debe considerarse el aumento de consumo energético y temperatura.
\end{itemize}

En resumen, aumentar la frecuencia del clock de la ESP32 tiene un impacto positivo y medible en el rendimiento, lo cual es especialmente relevante en aplicaciones críticas en tiempo.

