\subsubsection*{1. Uso cotidiano de la PC}
La PC se utiliza principalmente para desarrollo de software y ejecución de simulaciones. Los programas y entornos utilizados incluyen:
\begin{itemize}
    \item \textbf{Desarrollo de software:} Visual Studio Code, PyCharm, Eclipse.
    \item \textbf{Simulaciones:} MATLAB, Simulink, ANSYS.
    \item \textbf{Otros:} Navegación web con múltiples pestañas, edición de documentos en LaTeX.
\end{itemize}

\subsubsection*{2. Tareas y benchmarks representativos}

A continuación se presenta una tabla con algunas tareas frecuentes relacionadas con simulaciones y desarrollo de código, junto con el benchmark que mejor representa cada una.

\begin{center}
\begin{tabular}{|p{7cm}|p{7cm}|}
\hline
\textbf{Tarea} & \textbf{Benchmark representativo} \\
\hline
Compilar proyectos grandes en C/C++ & Geekbench o Cinebench (CPU multi-core) \\
\hline
Ejecución de simulaciones en MATLAB/Simulink & SPEC CPU2017 \\
\hline
Desarrollo de software en entornos pesados (e.g., PyCharm, Eclipse) & PassMark CPU \\
\hline
Análisis de datos y cálculos intensivos en Python & Geekbench (CPU single-core) \\
\hline
\end{tabular}
\end{center}

\subsubsection*{3. Reflexión final}
En términos generales, la PC cumple con los requerimientos diarios para el desarrollo de software y la ejecución de simulaciones. Sin embargo, se ha identificado que el rendimiento de la CPU puede convertirse en un cuello de botella durante la ejecución de simulaciones complejas y la compilación de proyectos grandes. 

Existe interés en evaluar y mejorar el rendimiento en estas tareas específicas, por lo que se considerará realizar actualizaciones de hardware, como la incorporación de un procesador más potente o la ampliación de la memoria RAM, para optimizar el rendimiento.
