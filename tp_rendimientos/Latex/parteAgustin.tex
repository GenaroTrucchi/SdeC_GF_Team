\subsubsection*{1. Uso cotidiano de la PC}
Describir brevemente para qué se usa la PC. Por ejemplo: desarrollo de software, diseño electrónico, juegos, edición de video, etc. Mencionar programas o entornos utilizados.

\subsubsection*{2. Tareas y benchmarks representativos}

A continuación se presenta una tabla con algunas tareas frecuentes y el benchmark que mejor representa cada una.

\begin{center}
\begin{tabular}{|p{7cm}|p{7cm}|}
\hline
\textbf{Tarea} & \textbf{Benchmark representativo} \\
\hline
Ejemplo: Compilar proyectos grandes en C/C++ & Geekbench o Cinebench (CPU multi-core) \\
\hline
Ejemplo: Uso de KiCad para diseño de PCBs & PassMark 2D Graphics (fluidez gráfica 2D) \\
\hline
Ejemplo: Navegación con muchas pestañas abiertas y uso de herramientas online & PCMark (rendimiento general en tareas de oficina/web) \\
\hline
\end{tabular}
\end{center}

\subsubsection*{3. Reflexión final}
Comentario sobre si la PC cumple con los requerimientos diarios, si algún componente se vuelve cuello de botella, y si hay interés en evaluar o mejorar el rendimiento en alguna tarea específica.
